\documentclass[a4paper,12pt]{report}
\usepackage[utf8]{inputenc}
\usepackage[french]{babel}
\usepackage{geometry}
\usepackage{xcolor}
\usepackage{listings}
\usepackage[utf8]{inputenc}
\usepackage[T1]{fontenc} 
\usepackage{lmodern}


\geometry{top=1in, bottom=1in, left=1in, right=1in}

% Couleurs personnalisées
\definecolor{myblue}{RGB}{50, 100, 200}
\definecolor{mygreen}{RGB}{50, 150, 50}
\definecolor{myred}{RGB}{200, 50, 50}

\lstset{
    basicstyle=\ttfamily\footnotesize,
    breaklines=true,
    frame=single,
    captionpos=b
}

\title{\textbf{Structure du Projet de Gestion des Employés}}
\author{Fatim ezzahrae AKNIZ}

\begin{document}

\maketitle

\section*{\textcolor{myblue}{Introduction}}

Le dossier présente les 4 parties du projets: Employees, Congées, Importer/exporter, se connecter/ajouter compte. Il est conçue avec le modèle 
MVC et DAO, d'où il se compose de 5 packages: Model, View, Controlleur, DAO, package par défaut(Main).
\vspace{0.3cm}

\textbf{Note:} La classe DBConnection est une classe centralisé pour les quatres parties et se situe dans le package DAO

\vspace{0.3cm}
\begin{lstlisting}[language=Java, caption=DAO implementation]
public static Connection getConnection() {
    if (conn == null) { // Check if connection is null
        try {
            // Load the JDBC driver
            Class.forName("com.mysql.cj.jdbc.Driver");
            // Establish the connection
            conn = DriverManager.getConnection(url, user, password);
        } catch (ClassNotFoundException | SQLException e) {
            e.printStackTrace();
            throw new RuntimeException("Error lors de la connexion");
        }
    }
    return conn;
}
\end{lstlisting}

\newpage

\section*{\textcolor{myblue}{1. Interface generique}}
Cette interface définit un ensemble de méthodes communes à l'entité Employe et Holiday, permettant ainsi de centraliser et de standardiser leur implémentation.
\begin{lstlisting}[language=Java, caption=le corps de l'interface]
public interface GenericDAOI<T> {

    Object[][] findById(int employeeID) throws SQLException, Exception;
    
    void add(T employe) throws SQLException, Exception;
    
    void update(T employe, int emp_id) throws SQLException, Exception;
    
    void delete(T employe) throws SQLException, Exception;
    
    Object[][] findAll() throws SQLException, Exception;


}
\end{lstlisting}

\newpage

\section*{\textcolor{myblue}{2. Employe}}

\subsection*{\textcolor{mygreen}{2.1 Employe.java}}
Cette classe constiyue les contructeurs, les getters et les setters de l'entité Employe:
\begin{lstlisting}[language=Java, caption=Constructeur de l'employe]
public Employe(String nom, String prenom, String email, String tel, int salaire, Role role, Poste poste) {
    this.email = email;
    this.nom = nom;
    this.prenom = prenom;
    this.salaire = salaire;
    this.tel = tel; // Updated
    this.role = role;
    this.poste = poste;
    this.holiday = 25;
}
\end{lstlisting}

\subsection*{\textcolor{mygreen}{2.2 EmployeDAOImpl.java}}
Cette classe présente la gestion des données dans la base de données: Ajout, edit, et suppression
\begin{lstlisting}[language=Java, caption=La methodes pour ajouter un employe]
@Override
public void add(Employe employe) throws Exception{
            
    String sql = "Insert Into Employe (nom, prenom, email, tel, salaire, Poste, Role ) values (?,?,?,?,?,?,?)";
    
    try(PreparedStatement stmt = conn.prepareStatement(sql)) {
        
        stmt.setString(1, employe.getNom());
        stmt.setString(2, employe.getPrenom());
        stmt.setString(3, employe.getEmail());
        stmt.setString(4, employe.getTel());
        stmt.setInt(5, employe.getSalaire());
        stmt.setString(6, employe.getPoste().name());
        stmt.setString(7, employe.getRole().name());
        stmt.executeUpdate();
        
    } catch (SQLException e) {
        throw new Exception("Error in add Emp: " + e.getMessage(), e);		
    }
}
\end{lstlisting}

\subsection*{\textcolor{mygreen}{2.3 EmployeModel.java}}
Cette classe encapsule les methodes de la classe EmployeDAO pour une gestion des données sécurisées, en ajoutant ainsi les regles de metiers 
\begin{lstlisting}[language=Java, caption=Exemple de quelques methodes]
// Validate employee data
private void validateEmployee(Employe employe) throws Exception {
    if (employe == null) {
        throw new IllegalArgumentException("Employee cannot be null.");
    }
    if (employe.getNom() == null || employe.getNom().trim().isEmpty()) {
        throw new IllegalArgumentException("Employee name cannot be null or empty.");
    }
    if (employe.getPrenom() == null || employe.getPrenom().trim().isEmpty()) {
        throw new IllegalArgumentException("Employee first name cannot be null or empty.");
    }
    if (employe.getEmail() == null || !employe.getEmail().matches("^[A-Za-z0-9+_.-]+@[A-Za-z0-9.-]+$")) {
        throw new IllegalArgumentException("Invalid email address.");
    }
    if (employe.getTel() == null || 
        !employe.getTel().matches("^0[67][0-9]{8}$")) {
        throw new IllegalArgumentException("Phone number must be 10 digits and start with 06 or 07.");
    }
    if (employe.getSalaire() <= 0) {
        throw new IllegalArgumentException("Salary must be a positive number.");
    }
}


// Get an employee by ID
public Employe getEmployeeById(int emp_id) throws Exception {
    if (emp_id <= 0) {
        throw new IllegalArgumentException("Invalid employee ID.");
    }

    try {
        Employe employe = employeDAO.getEmployeById(emp_id);
        if (employe == null) {
            throw new Exception("Employee not found with ID: " + emp_id);
        }
        return employe;
    } catch (SQLException e) {
        throw new Exception("Error while retrieving employee: " + e.getMessage(), e);
    }
}
\end{lstlisting}

\subsection*{\textcolor{mygreen}{2.4 EmployeView.java}}
Cette classe constitue la vue que l'utulisateur travaille avec. En plus, il se compose des methodes pour communiquer les données avec le controlleur, et les faire mise à jour
\begin{lstlisting}[language=Java, caption=le contructeur de la classe]
public EmployeeView() {
    rolesCombo = new JComboBox<>(Role.values());
    postesCombo = new JComboBox<>(Poste.values());
    
    tableModel = new DefaultTableModel(new Object[][] {}, tableColumns);
    tbl = new JTable(tableModel);
    tbl.setAutoResizeMode(JTable.AUTO_RESIZE_OFF);
    tbl.setFillsViewportHeight(true);
    
    
    
    JPanel data = new JPanel();
    data.setLayout(new BorderLayout());
    
    JPanel entries = new JPanel();
    entries.setLayout(new GridLayout(7,2));
    entries.add(new JLabel("Nom:"));
    entries.add(nom);
    entries.add(new JLabel("Prenom:"));
    entries.add(prenom);
    entries.add(new JLabel("Email:"));
    entries.add(email);
    entries.add(new JLabel("Telephone:"));
    entries.add(tel);
    entries.add(new JLabel("Salaire:"));
    entries.add(salaire);
    entries.add(new JLabel("Role:"));
    entries.add(rolesCombo);
    entries.add(new JLabel("Poste:"));
    entries.add(postesCombo);
    
    data.add(entries, BorderLayout.CENTER);
    
    JScrollPane scroll = new JScrollPane(tbl);	
    tbl.setPreferredScrollableViewportSize(new Dimension(800, 200));  // Adjust as needed
    applyTableColumnWidths();
    JPanel table = new JPanel();
    table.add(scroll);
    data.add(table, BorderLayout.SOUTH);
    
    add(data, BorderLayout.CENTER);
    
    JPanel buttons = new JPanel();
    buttons.setLayout(new FlowLayout());
    buttons.add(addButton);
    buttons.add(updateButton);
    buttons.add(deleteButton);
    buttons.add(listButton);
    buttons.add(findIdButton);
    buttons.add(importButton);
    buttons.add(exportButton);
    buttons.add(createUserAcc);
    
    add(buttons, BorderLayout.SOUTH);
}
\end{lstlisting}

\subsection*{\textcolor{mygreen}{2.5 EmployeController.java}}
Cette classe sert à collecter les données de la vue, et les manipuler en utilisant les methodes de EmployeModel, tout ça sous en réponse d'un evenement créé par l'utulisateur
\begin{lstlisting}[language=Java, caption=action de la boutton supprimer]
view.getDeleteButton().addActionListener(new ActionListener() {
    @Override
    public void actionPerformed(ActionEvent e) {
        if (selectedEmployeeId == -1) {
            JOptionPane.showMessageDialog(view, "Please select an employee to delete." , "Error", JOptionPane.ERROR_MESSAGE);
            return;
        }

        int confirm = JOptionPane.showConfirmDialog(view, 
            "Are you sure you want to delete this employee?", 
            "Confirm Delete", 
            JOptionPane.YES_NO_OPTION);

        if (confirm == JOptionPane.YES_OPTION) {
            try {
                Employe employe = employeModel.getEmployeeById(selectedEmployeeId);
                employeModel.deleteEmployee(employe);
                selectedEmployeeId = -1;
                Object[][] data = employeModel.getAllEmployees();
                view.refreshTable(data);
                JOptionPane.showMessageDialog(view, "Employee deleted successfully!", "Success", JOptionPane.INFORMATION_MESSAGE);
            } catch (Exception ex) {
                JOptionPane.showMessageDialog(view, "Failed to delete employee: " + ex.getMessage(), "Error", JOptionPane.ERROR_MESSAGE);
            }
        }
    }
});
\end{lstlisting}


\textbf{Note:} La description des classes est la même pour l'entité Holiday.
\newpage
\section*{\textcolor{myblue}{3. Holiday}}

\subsection*{\textcolor{mygreen}{3.1 Holiday.java}}
\begin{lstlisting}[language=Java, caption=Constructeurs de Holiday]
public Holiday(int id, String DDebut, String DFin, HolidayType type, String employe) {
    this.id = id;
    this.DDebut = DDebut;
    this.DFin = DFin;
    this.type = type;
    this.employe = employe;
}

public Holiday(String DDebut, String DFin, int idType, int idEmp) {
    this.DDebut = DDebut;
    this.DFin = DFin;
    this.idType = idType;
    this.idEmp = idEmp;
}
\end{lstlisting}

\subsection*{\textcolor{mygreen}{3.2 HolidayDAOImpl.java}}
\begin{lstlisting}[language=Java, caption=methode pour editer une holiday]
public void update(Holiday employe, int emp_id) throws SQLException {
    String sql = "SELECT id from typeholiday where type = ?";
    
    try(PreparedStatement stmt = conn.prepareStatement(sql)) {
        
        stmt.setString(1, employe.getType().toString());
        try(ResultSet rs = stmt.executeQuery()) {
            if(rs.next()) {
                idType = rs.getInt("id");
            } else {
                throw new SQLException("Type not found: " + employe.getType().toString());
            }
        }
        
    } catch (SQLException e) {
        e.printStackTrace();
    }
    
    String sql1 = "UPDATE holiday SET idEmp = ?, idType = ?, DDebut = ?, DFin = ? WHERE id = ?";
    try(PreparedStatement stmt = conn.prepareStatement(sql1)) {
        
        stmt.setInt(1, getIdEmp(employe.getEmploye()));
        stmt.setInt(2, idType);
        stmt.setString(3, employe.getDDebut());
        stmt.setString(4, employe.getDFin());
        stmt.setInt(5, emp_id);
        stmt.executeUpdate();
    } catch (SQLException e) {
        e.printStackTrace();
    }
    
}
\end{lstlisting}

\subsection*{\textcolor{mygreen}{3.3 HolidayModel.java}}
\begin{lstlisting}[language=Java, caption=Exemple de methodes avec regles de métier]
public int calculateHoliday(String DDate, String FDate, int idEmp) throws Exception {
    int EmpHoliday = dao.getEmployeeHoliday(idEmp);
    if (EmpHoliday == -1) {
        throw new Exception("Failed to retrieve remaining holiday balance.");
    }
    
    int updatedHoliday = HolidayPeriod(DDate, FDate, EmpHoliday);
    if (updatedHoliday < 0) {
        throw new Exception("Holiday period exceeds the permitted limit.");
    }
    return updatedHoliday;
}

public int getEmployeeHoliday(int idEmp) throws Exception {
    int EmpHoliday = dao.getEmployeeHoliday(idEmp);
    if (EmpHoliday == -1) {
        throw new Exception("Failed to retrieve remaining holiday balance.");
    }
    return EmpHoliday;
}
\end{lstlisting}

\subsection*{\textcolor{mygreen}{3.4 HolidayView.java}}
\begin{lstlisting}[language=Java, caption=La vue pour Holiday]
public HolidayView() {
    ArrayList<String> employes = dao.getEmployees();
    employees = new JComboBox<>(employes.toArray(new String[0]));
    
    ArrayList<HolidayType> types = dao.getHolidayType();
    typeHoliday = new JComboBox<>(types.toArray(new HolidayType[0]));
    
    JPanel data = new JPanel();
    data.setLayout(new BorderLayout());
    
    JPanel entries = new JPanel();
    entries.setLayout(new GridLayout(4,2));
    
    entries.add(new JLabel("Nom de l'employe:"));
    entries.add(employees);
    entries.add(new JLabel("Type:"));
    entries.add(typeHoliday);
    entries.add(new JLabel("Date de debut:"));
    entries.add(dateDebut);
    entries.add(new JLabel("Date de fin:"));
    entries.add(dateFin);
    
    data.add(entries, BorderLayout.CENTER);
    
    
    tableModel = new DefaultTableModel(dao.getHoliday(), tableColumns);
    tbl = new JTable(tableModel);
    tbl.setFillsViewportHeight(true);
    
    //hide the idEmp row
    hideEmpColumn();
    
    JScrollPane scroll = new JScrollPane(tbl);	
    tbl.setPreferredScrollableViewportSize(new Dimension(700, 150));  // Adjust as needed
    JPanel table = new JPanel();
    table.add(scroll);
    
    data.add(table, BorderLayout.SOUTH);
    
    add(data, BorderLayout.CENTER);
    
    JPanel buttons = new JPanel();
    buttons.setLayout(new FlowLayout());
    buttons.add(addButton);
    buttons.add(updateButton);
    buttons.add(deleteButton);
    buttons.add(importButton);
    buttons.add(exportButton);
    
    add(buttons, BorderLayout.SOUTH);
    
}
\end{lstlisting}

\subsection*{\textcolor{mygreen}{3.5 HolidayController.java}}
\begin{lstlisting}[language=Java, caption=Action de la boutton Ajouter]
//add a new employee
view.getAddButton().addActionListener(new ActionListener() {
        
    @Override
    public void actionPerformed(ActionEvent e) {
        
        if (!view.validateFields()) return;			               
        
        Employe employe = extractEmployeeFromFields();
        
        try {
            employeModel.addEmployee(employe);
            Object[][] data = employeModel.getAllEmployees();
            view.refreshTable(data);
            JOptionPane.showMessageDialog(view, "Employee added successfully!", "Success", JOptionPane.INFORMATION_MESSAGE);
        } catch (Exception ex) {
            JOptionPane.showMessageDialog(view, "Failed to add employee: " + ex.getMessage(), "Error", JOptionPane.ERROR_MESSAGE);
        }
    }						
});
\end{lstlisting}

\section*{\textcolor{myblue}{4. JTabbedPane}}
En ce qui concerne cette classe, il est ajouté pour combiner les deux vues (Employe, Holiday) dans une seule fenêtre 
avec une navigation simple et facile 
\begin{lstlisting}[language=java, caption=le corps de la classe]
public class MainView extends JFrame{

	private JTabbedPane tabbedPane;
	
	public MainView(EmployeeView viewEmp, HolidayView viewHol) {
		
		setTitle("Application de gestion");
		setSize(850, 550);
		setDefaultCloseOperation(JFrame.EXIT_ON_CLOSE);
		setLocationRelativeTo(null);
		
		tabbedPane = new JTabbedPane();
		
		tabbedPane.addTab("Employees", viewEmp);
		tabbedPane.addTab("Congees", viewHol);
		
		add(tabbedPane);
		
		setVisible(true);
		
	}
		
}
\end{lstlisting}
\newpage

\section*{\textcolor{myblue}{5. Importer/Exporter}}
Pour cette option deux bouttons sont ajoutées à la vue de l'employée pour importer ou exporter ses données. En plus, d'autres 
deux bouttons sont ajoutées à la vue des congées pour importer ou exporter les informations des employées ainsi que ses congées déclarés.
La gestion de ces boutton est faite dans les classes de l'employer et holiday.
\subsection{\textcolor{mygreen}{5.1 Interface générique}}
Une interface générique est conçue pour ces deux fonctionnalitées et implémentée par Employe et Holiday pour centraliser les methodes à utiliser
\begin{lstlisting}[language=java, caption={Le corps de l'interface}]
public interface DataImportExport<T> {

    void importData(String fileName) throws IOException;
    
    void exportData(String fileName, List<T> data) throws IOException;

}
\end{lstlisting}
\subsection{\textcolor{mygreen}{5.2 Données employées}}
\begin{lstlisting}[language=java, caption={le methode pour exporter les données des employées}]
public void exportData(String filePath, List<Employe> data) throws IOException {
    
    try(BufferedWriter writer = new BufferedWriter(new FileWriter(filePath))) {
        
        writer.write("Nom, Prenom, Email, Tel, Salaire, Poste, Role");
        writer.newLine();
        for (Employe employe : data) {
            String line = String.format("%s, %s, %s, %s, %d, %s, %s",
                    employe.getNom(),
                    employe.getPrenom(),
                    employe.getEmail(),
                    employe.getTel(),
                    employe.getSalaire(),
                    employe.getPosteSt(),
                    employe.getRoleSt()
                    );
            writer.write(line);
            writer.newLine();
        }
        
    } 
    
}
\end{lstlisting} 
\subsection{\textcolor{mygreen}{5.3 Données employées avec congées}}
\begin{lstlisting}[language=java, caption={La methode pour importer les données des employées aves ses congées}]
public void importData(String filePath) throws IOException {
    String sql = "Insert Into Employe (nom, prenom, email, tel, salaire, Poste, Role ) values (?,?,?,?,?,?,?)";

    try(BufferedReader reader = new BufferedReader(new FileReader(filePath)); PreparedStatement pstmt = conn.prepareStatement(sql)) {
        String line;		    
        while((line = reader.readLine()) != null) {
            System.out.println("First Line: " + line);
            if (line == null) {
                System.out.println("The file is empty.");
            }
            String[] data = line.split(",");
            System.out.println("Data length: " + data.length); 
            if(data.length == 11) {
                System.out.println("Processing line: " + line);
                pstmt.setString(1, data[0].trim());
                pstmt.setString(2, data[1].trim());
                pstmt.setString(3, data[2].trim());
                pstmt.setString(4, data[3].trim());
                pstmt.setString(5, data[4].trim());
                pstmt.setString(6, data[5].trim());
                pstmt.setString(7, data[6].trim());
                pstmt.addBatch();
                Holiday holiday = new Holiday(
                        data[7].trim(),
                        data[8].trim(),
                        HolidayType.valueOf(data[9].trim()),
                        data[10].trim()
                        );
                add(holiday);
            } else {
                System.out.println("Skipping line due to incorrect number of data fields");
            }
        }
        pstmt.executeBatch();
        System.out.println("Holiday imported successfully!");            
    } catch (IOException | SQLException e) {
        e.printStackTrace();    
    }		
}
\end{lstlisting}
\newpage

\section*{\textcolor{myblue}{6. Se connecter/Ajouter compte}}
\subsection{\textcolor{mygreen}{6.1 Se connecter}}
Cette partie est traité independamment avec tous les classes appropriée(Model, view, cntrolleur...), et est conçue d'apparaître en premier et alors
si les identifiants sont correctes passer à la fenêtre de gestion des employées et congées avec les droits spécifiés en fonction de son role sinon un message d'erreur apparaîtra(ceci est traité dans le main).
\begin{lstlisting}[language=java, caption={Le processus declenché après la validation des identifiants par l'utilisateur}]
view.getLoginButton().addActionListener(new ActionListener() {
    @Override
    public void actionPerformed(ActionEvent e) {
        String user = view.getUser().getText();
        String pass = view.getPass().getText();

        try {
            // Validate user and get the role
            userRole = model.validateUserAndGetRole(user, pass);
            
            if (userRole != null) {
                // Pass the role to EmployeeView and HolidayView
                employeeView.setUserRole(userRole);
                holidayView.setUserRole(userRole);
                
                // Notify login success
                if (loginSuccessListener != null) {
                    loginSuccessListener.run(); // Notify that login is successful
                }
            } else {
                JOptionPane.showMessageDialog(view, "Invalid login credentials!", "Error", JOptionPane.ERROR_MESSAGE);
            }
            
        } catch (Exception ex) {
            JOptionPane.showMessageDialog(view, "Login failed: " + ex.getMessage(), "Error", JOptionPane.ERROR_MESSAGE);
        }
    }
});
\end{lstlisting}
\subsection{\textcolor{mygreen}{6.2 Ajouter Compte utilisateur}}
Cette fonction est dédiée au role du Manager seulement, il le permet de choisir depuis un JComboBox l'employe auquel il veut créer un compte
puis entrer le nom d'utilisateur, et le most de passe associé à ce compte. Ceci permet de créer une nouvelle ligne dans la table login dans la base 
de données avec un username et un mot de passe hashé pour des raisons des sécurit.
\begin{lstlisting}[language=java, caption={Action de la boutton créer compte}]
view.getAddAccButton().addActionListener(new ActionListener() {

    @Override
    public void actionPerformed(ActionEvent e) {
        ArrayList<Employe> emps;
        try {
            emps = employeModel.ListEmployes_NomPre();
            // Pass a callback to handle the data submitted by the user
            view.createUserAcc(emps, (data) -> {
                // This is the callback that will be invoked when the user submits the form
                String employee = data.get("employee");
                String username = data.get("username");
                String password = data.get("password");
                
                int idEmp = employeModel.getEmpId(employee);
                String hashedPassword = BCrypt.hashpw(password, BCrypt.gensalt());
                // Call the model method to add the user to the database
                boolean success = employeModel.addUser(idEmp, username, hashedPassword);

                if (success) {
                    JOptionPane.showMessageDialog(view, "User added successfully!");
                } else {
                    JOptionPane.showMessageDialog(view, "Failed to add user.");
                }
                
                // Here you can call methods on the model to save the data
                // For example: employeModel.addUserAccount(employee, username, password);
            });
        } catch (Exception e1) {
            // TODO Auto-generated catch block
            e1.printStackTrace();
        }
    }		
});
\end{lstlisting}
\newpage

\section*{\textcolor{myblue}{7. Main}}
La classe Main joue un rôle central dans l'application. Elle est responsable de créer les instances nécessaires et
 d'appeler les constructeurs appropriés. De plus, elle gère l'enchaînement logique entre l'affichage 
 de la fenêtre de connexion et celle de gestion, assurant ainsi une expérience utilisateur fluide.
\begin{lstlisting}[language=java, caption={Main}]
public class Main {
    public static void main(String[] args) {
        // Initialize login components
        loginDAOImpl dao = new loginDAOImpl();
        loginModel loginModel = new loginModel(dao);
        loginView loginView = new loginView();
        
        // Initialize the DAO for Employee
        EmployeeDAOImpl employeeDAO = new EmployeeDAOImpl();
        EmployeModel employeModel = new EmployeModel(employeeDAO);
        EmployeeView employeeView = new EmployeeView(); // Create employee view
        
        // Initialize the DAO for Holiday
        HolidayDAOImpl holidayDAO = new HolidayDAOImpl();
        HolidayModel holidayModel = new HolidayModel(holidayDAO);
        HolidayView holidayView = new HolidayView(); // Create holiday view

        // Initialize the controllers
        loginController loginController = new loginController(loginModel, loginView, employeeView, holidayView);
        EmployeeController employeeController = new EmployeeController(employeModel, employeeView);
        HolidayController holidayController = new HolidayController(holidayModel, holidayView);
        
        // Show the login view first
        loginView.setVisible(true);
        
        // Initialize the MainView after successful login
        // Here, you'll switch to MainView after login success
        loginController.setLoginSuccessListener(new Runnable() {
            @Override
            public void run() {
                // Hide the login view
                loginView.setVisible(false);
                
                // Show the MainView with EmployeeView and HolidayView in tabs
                new MainView(employeeView, holidayView).setVisible(true);
            }
        });
    }
}
\end{lstlisting}
\end {document}